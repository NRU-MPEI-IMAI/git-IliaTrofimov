\documentclass{article}
\usepackage[T2A]{fontenc}
\usepackage[utf8]{inputenc}
\usepackage[russian]{babel}
\usepackage{textcomp}
\usepackage{color}
\usepackage{xspace}
\usepackage{multirow}
\usepackage{amsmath,amsfonts,amsthm,amssymb,amsbsy,amstext,amscd,amsxtra,multicol}
\usepackage{indentfirst}
\usepackage{verbatim}
\usepackage[left=2cm, right=2cm, top=2cm, bottom=2cm, bindingoffset=0cm]{geometry}
\usepackage[pdf]{graphviz}
\usepackage{morewrites}


\usepackage{xpatch}
\makeatletter
\newcommand*{\addFileDependency}[1]{% argument=file name and extension
  \typeout{(#1)}
  \@addtofilelist{#1}
  \IfFileExists{#1}{}{\typeout{No file #1.}}
}
\makeatother
\xpretocmd{\digraph}{\addFileDependency{#2.dot}}{}{}



\begin{document}
    \vbox{%
        \hfill%
        \vbox{%
            \hbox{\large Выполнил студент группы А-05-19 \par}%
            \hbox{\large Трофимов Илья Сергеевич \par}%
            \hbox{\large 5 апреля 2022 г.\par}%
        }%
    } 
    
    \begin{center}
        {\LARGE  \textbf{Домашняя работа №1}\par}
        {\Large по Теоретическим моделям вычислений\par}
    \end{center}
    
    \section*{Задание 1}
    Построить конечные автоматы\footnote{Так как библиотека graphviz для \LaTeX по неизвестной мне причине, не может использовать строки в кавыках в качестве имён узлов и не может распознавать имена, состоящие из расположенных слитно цифр и букв, то названия узлов в некоторых приведённых здесь графах будут иметь длинные числовые наименования. }, распознающие следующие языки:
    \begin{enumerate}
        \item \(L_1=\{\omega\in\{a,b,c\}^* : |\omega|_c = 1 \} \)
        \begin{center}
            \digraph[scale=0.5]{g11}{
                node [shape = none]; 0;
                node [shape = doublecircle]; 2;
    	        node [shape = circle];
                rankdir=LR; 
                0 -> 1;
                1 -> 2 [label = "c"];
                1 -> 1 [label = "a,b"];
                2 -> 2 [label = "a,b"];
            }   
        \end{center}
        \item \(L_2=\{\omega\in\{a,b\}^* : |\omega|_a \leqslant 2, |\omega|_b \geqslant 2 \} \) \\
        Рассмотрим автоматы \(A=\{\omega\in\{a,b\}^* : |\omega|_a \leqslant 2 \} \) и \(B=\{\omega\in\{a,b\}^* : |\omega|_B \geqslant 2 \} \), распознающиие каждое условие по отдельности:
        \begin{center}
            \digraph[scale=0.5]{g121}{
                node [shape = doublecircle]; 1 2 3;
                node [shape = none]; 0;
                rankdir=LR; 
                0 -> 1;
                1 -> 2 [label = "a"];
                2 -> 3 [label = "a"];
                1 -> 1 [label = "b"];
                2 -> 2 [label = "b"];
                3 -> 3 [label = "b"];
            }
            \digraph[scale=0.5]{g122}{
                node [shape = doublecircle]; 3;
                node [shape = none]; 0;
                node [shape = circle]; 1 2;
                rankdir=LR; 
                0 -> 1;
                1 -> 2 [label = "b"];
                2 -> 3 [label = "b"];
                1 -> 1 [label = "a"];
                2 -> 2 [label = "a"];
                3 -> 3 [label = "a,b"];
            }
        \end{center}
        Тогда \(L_2=A \times B\). Терминальными состояниями в \(L_2\) будут вершины 13, 23 и 33. Теперь выпишем переходы для произведения автоматов в виде таблицы:
        \begin{center}
            \begin{tabular}{ |c|c|c|c| } 
                \hline
                \(A\) & \(B\) & переход по \(a\) & переход по \(b\) \\
                \hline\hline
                1 & 1 & 21 & 12 \\
                \hline
                2 & 2 & 32 & 23 \\
                \hline
                3 & 3 & - & 33 \\
                \hline
                1 & 2 & 22 & 13 \\
                \hline
                2 & 3 & 33 & 23 \\
                \hline
                3 & 1 & - & 32 \\
                \hline
                1 & 3 & 23 & 13 \\
                \hline
                2 & 1 & 31 & 22 \\
                \hline
                3 & 2 & - & 33 \\
                \hline
                \end{tabular}
            \end{center}
        После прямого произведения двух автоматов получим окончательный ответ:
        \begin{center}
            \digraph[scale=0.5]{g123}{
                node [shape = none]; 0;
                node [shape = doublecircle]; 13 23 33;
    	        node [shape = circle];
                rankdir=LR; 
                0 -> 11;
                11 -> 21 [label = "a"];
                11 -> 12 [label = "b"]; 
                22 -> 32 [label = "a"];
                22 -> 23 [label = "b"];
                33 -> 33 [label = "b"];
            	12 -> 22 [label = "a"];
                12 -> 13 [label = "b"];
            	23 -> 33 [label = "a"];
                23 -> 23 [label = "b"];
                31 -> 32 [label = "b"];
            	13 -> 23 [label = "a"];
                13 -> 13 [label = "b"];
            	21 -> 31 [label = "a"];
                21 -> 22 [label = "b"];
                32 -> 33 [label = "b"];
            }
        \end{center}
        
        \item \(L_3=\{\omega\in\{a,b\}^*:|\omega|_a \neq |\omega|_b \}\) \\
        Этот язык нельзя описать с помощью ДКА, т.к. для описания языка необходимо запоминать количество символов одного типа, что ДКА сделать не может.
        \item \(L_4=\{\omega\in\{a,b\}^* : \omega \omega = \omega \omega \omega \} \) \\
        Очевидно, что такой язык описывает только пустые слова:
        \begin{center}
            \digraph[scale=0.5]{g14}{
                node [shape = none]; 0;
                node [shape = doublecircle]; 1;
                rankdir=LR; 
                0 -> 1;
            }  
        \end{center}
    \end{enumerate}
    
    \section*{Задание 2}
    Построить конечные автоматы, распознающие слудеющие языки, используя прямое произведение:
    \begin{enumerate}
        \item \(L_1=\{\omega\in\{a,b\}^* : |\omega|_a \geqslant 2 \wedge |\omega|_b
        \geqslant 2 \} \) \\
        Рассмотрим автоматы \(A=\{\omega\in\{a,b\}^* : |\omega|_a \geqslant 2 \} \) и \(B=\{\omega\in\{a,b\}^* : |\omega|_b \geqslant 2 \} \), распознающиие каждое условие по отдельности:
        \begin{center}
            \digraph[scale=0.5]{g211}{
                node [shape = none]; 0;
                node [shape = doublecircle]; 3;
                node [shape = circle]; 1 2;
                rankdir=LR; 
                0 -> 1;
                1 -> 2 [label = "a"];
                2 -> 3 [label = "a"];
                1 -> 1 [label = "b"];
                2 -> 2 [label = "b"];
                3 -> 3 [label = "a,b"];
            }
            \digraph[scale=0.5]{g212}{
                node [shape = none]; 0;
                node [shape = doublecircle]; 3;
                node [shape = circle]; 1 2;
                rankdir=LR; 
                0 -> 1;
                1 -> 2 [label = "b"];
                2 -> 3 [label = "b"];
                1 -> 1 [label = "a"];
                2 -> 2 [label = "a"];
                3 -> 3 [label = "a,b"];
            }    
        \end{center}
        Тогда \(L_1=A \times B\), имеем \(\Sigma=\{a,b\}\), \(s=11\) и \(T=\{33\}\). Теперь выпишем переходы для произведения автоматов в виде таблицы:
        \begin{center}
            \begin{tabular}{ |c|c|c|c| } 
                \hline
                \(A\) & \(B\) & переход по \(a\) & переход по \(b\) \\
                \hline
                1 & 1 & 21 & 12 \\
                \hline
                2 & 2 & 32 & 23 \\
                \hline
                3 & 3 & 33 & 33 \\
                \hline
                1 & 2 & 22 & 13 \\
                \hline
                2 & 3 & 33 & 23 \\
                \hline
                3 & 1 & 31 & 32 \\
                \hline
                1 & 3 & 23 & 13 \\
                \hline
                2 & 1 & 31 & 22 \\
                \hline
                3 & 2 & 32 & 33 \\
                \hline
            \end{tabular}
        \end{center}
        После прямого произведения двух автоматов получим окончательный ответ:
        \begin{center}
            \digraph[scale=0.5]{g213}{
                node [shape = none]; 0;
                node [shape = doublecircle]; 33;
    	        node [shape = circle];
                rankdir=LR; 
                0 -> 11;
                11 -> 21 [label = "a"];
                11 -> 12 [label = "b"]; 
                22 -> 32 [label = "a"];
                22 -> 23 [label = "b"];
                33 -> 33 [label = "a,b"];
                
            	12 -> 22 [label = "a"];
                12 -> 13 [label = "b"];
            	23 -> 33 [label = "a"];
                23 -> 23 [label = "b"];
                31 -> 31 [label = "a"];
                31 -> 32 [label = "b"];
                
            	13 -> 23 [label = "a"];
                13 -> 13 [label = "b"];
            	21 -> 31 [label = "a"];
                21 -> 22 [label = "b"];
                32 -> 32 [label = "a"];
                32 -> 33 [label = "b"];
            }
        \end{center}
        
        \item \(L_2=\{\omega \in\{a,b\}^* : |\omega| \geqslant 3 \wedge |\omega| \text{ нечётное} \} \) \\
        Рассмотрим автоматы \(A=\{\omega \in\{a,b\}^* : |\omega| \geqslant 3\} \) и \(B=\{\omega \in\{a,b\}^* : |\omega| \text{ нечётное} \} \):
        \begin{center}
            \digraph[scale=0.5]{g221}{
                node [shape = none]; 0;
                node [shape = doublecircle]; 4;
                node [shape = circle]; 1 2, 3;
                rankdir=LR; 
                0 -> 1;
                1 -> 2 [label = "a, b"];
                2 -> 3 [label = "a, b"];
                3 -> 4 [label = "a, b"];
                4 -> 4 [label = "a, b"];
            }
            \digraph[scale=0.5]{g222}{
                node [shape = none]; 0;
                node [shape = doublecircle]; 2;
                node [shape = circle]; 1;
                rankdir=LR; 
                0 -> 1;
                1 -> 2 [label = "a, b"];
                2 -> 1 [label = "a, b"];
            }    
        \end{center}
        Тогда \(L_2=A \times B\), имеем \(\Sigma=\{a,b\}\), \(s=11\) и \(T=\{33\}\). Переходы для произведения автоматов:
        \begin{center}
            \begin{tabular}{ |c|c|c| } 
                \hline
                \(A\) & \(B\) & переход по \(a\) или \(b\) \\
                \hline\hline
                1 & 1 & 22 \\
                \hline
                2 & 1 & 32 \\
                \hline
                3 & 1 & 42 \\
                \hline
                4 & 1 & 42 \\
                \hline
                1 & 2 & 21 \\
                \hline
                2 & 2 & 31 \\
                \hline
                3 & 2 & 41 \\
                \hline
                4 & 2 & 41 \\
                \hline
            \end{tabular}
        \end{center}
        После прямого произведения двух автоматов получим окончательный ответ:
        \begin{center}
            \digraph[scale=0.5]{g223}{
                node [shape = none]; 0;
                node [shape = doublecircle]; 42;
    	        node [shape = circle];
                rankdir=LR; 
                0 -> 11;
                11 -> 22 [label = "a,b"];
                21 -> 32 [label = "a,b"];
                31 -> 42 [label = "a,b"];
                41 -> 42 [label = "a,b"];
                12 -> 21 [label = "a,b"];
                22 -> 31 [label = "a,b"];
                32 -> 41 [label = "a,b"];
                42 -> 41 [label = "a,b"];
            }
        \end{center}
        ДКА можно упростить, т.к. невозможно попасть в узлы 12, 21 и 32:
        \begin{center}
            \digraph[scale=0.5]{g224}{
                node [shape = none]; 0;
                node [shape = doublecircle]; 42;
    	        node [shape = circle];
                rankdir=LR; 
                0 -> 11;
                11 -> 22 [label = "a,b"];
                31 -> 42 [label = "a,b"];
                41 -> 42 [label = "a,b"];
                22 -> 31 [label = "a,b"];
                42 -> 41 [label = "a,b"];
            }
        \end{center}
        С другой стороны, описать данный язык можно с помощью более компактного автомата, созданного "вручную":
        \begin{center}
            \digraph[scale=0.5]{g225}{
                node [shape = none]; 0;
                node [shape = doublecircle]; 4;
    	        node [shape = circle];
                rankdir=LR; 
                0 -> 1;
                1 -> 2 [label = "a,b"];
                2 -> 3 [label = "a,b"];
                3 -> 4 [label = "a,b"];
                4 -> 3 [label = "a,b"];
            }
        \end{center}
        
        \item \(L_3=\{\omega \in\{a,b\}^* : |\omega|_a \text{ чётно} \wedge |\omega|_b \text{ кратно } 3 \} \) \\
        Рассмотрим автоматы \(A=\{\omega \in\{a,b\}^* : |\omega|_a \text{ чётно} \} \) и \(B=\{\omega \in\{a,b\}^* : |\omega|_b \text{ кратно } 3 \} \):
        \begin{center}
            \digraph[scale=0.5]{g231}{
                node [shape = none]; 0;
                node [shape = doublecircle]; 1;
    	        node [shape = circle];
                rankdir=LR; 
                0 -> 1;
                1 -> 2 [label = "a"];
                2 -> 1 [label = "a"];
                1 -> 1 [label = "b"];
                2 -> 2 [label = "b"];
            }
            \digraph[scale=0.5]{g232}{
                node [shape = none]; 0;
                node [shape = doublecircle]; 1;
    	        node [shape = circle];
                rankdir=LR; 
                0 -> 1;
                1 -> 1 [label = "a"];
                2 -> 2 [label = "a"];
                3 -> 3 [label = "a"];
                1 -> 2 [label = "b"];
                2 -> 3 [label = "b"];
                3 -> 1 [label = "b"];
            }
        \end{center}
        Тогда \(L_3=A \times B\), имеем \(\Sigma=\{a,b\}\), \(s=11\) и \(T=\{11\}\). Переходы для произведения автоматов:
        \begin{center}
            \begin{tabular}{ |c|c|c|c| } 
                \hline
                \(A\) & \(B\) & переход по \(a\) & переход по \(b\) \\
                \hline\hline
                1 & 1 & 21 & 12 \\
                \hline
                2 & 2 & 12 & 23 \\
                \hline
                1 & 2 & 22 & 13 \\
                \hline
                2 & 3 & 13 & 21 \\
                \hline
                1 & 3 & 23 & 11 \\
                \hline
                2 & 1 & 11 & 22 \\
                \hline
            \end{tabular}
        \end{center}
        После прямого произведения двух автоматов получим окончательный ответ:
        \begin{center}
            \digraph[scale=0.5]{g233}{
                node [shape = none]; 0;
                node [shape = doublecircle]; 11;
    	        node [shape = circle];
                rankdir=LR; 
                0 -> 11;
                11 -> 21 [label = "a"];
                11 -> 12 [label = "b"];
                22 -> 12 [label = "a"];
                22 -> 23 [label = "b"];
                12 -> 22 [label = "a"];
                12 -> 13 [label = "b"];
                23 -> 13 [label = "a"];
                23 -> 21 [label = "b"];
                13 -> 23 [label = "a"];
                13 -> 11 [label = "b"];
                21 -> 11 [label = "a"];
                21 -> 22 [label = "b"];
            }
        \end{center}
        
        \item \(L_4= \neg L_3\) \\
        Имеем, \(T_4= Q_3 \setminus T_3 = \{ 12, 13, 21, 22, 23 \} \), тогда можно легко построить ДКА:
        \begin{center}
            \digraph[scale=0.5]{g24}{
                node [shape = none]; 0;
                node [shape = doublecircle]; 12 13 21 22 23;
    	        node [shape = circle]; 11;
                rankdir=LR; 
                0 -> 11;
                11 -> 21 [label = "a"];
                11 -> 12 [label = "b"];
                22 -> 12 [label = "a"];
                22 -> 23 [label = "b"];
                12 -> 22 [label = "a"];
                12 -> 13 [label = "b"];
                23 -> 13 [label = "a"];
                23 -> 21 [label = "b"];
                13 -> 23 [label = "a"];
                13 -> 11 [label = "b"];
                21 -> 11 [label = "a"];
                21 -> 22 [label = "b"];
            }
        \end{center}
        
        \item \(L_5 = L_2 \setminus L_3\) \\
        Для построения этого автомата используем упрощённую версию автомата \(L_2\), которая была получена в пункте 2 этого задания. Для удобства дальнейших преобразований, перенумеруем названия узлов графа \(\neg L_3\):
        \begin{center}
            \digraph[scale=0.5]{g252}{
            	rankdir=LR; 
            	node [shape=none]; 0;
            	node [shape=doublecircle]; 2 3 4 5 6;
            	node [shape=circle]; 1;
            	0 -> 1;
            	1 -> 2 [label = "b"];
            	1 -> 4 [label = "a"];
            	2 -> 3 [label = "b"];
            	2 -> 5 [label = "a"];
            	3 -> 1 [label = "b"];
            	3 -> 6 [label = "a"];
                4 -> 5 [label = "b"];
            	4 -> 1 [label = "a"];
            	5 -> 6 [label = "b"];
            	5 -> 2 [label = "a"];
            	6 -> 4 [label = "b"];
            	6 -> 3 [label = "a"];
            }
        \end{center}
        
        Так как \(L_5= L_2 \setminus L_3 = L_2 \cap \neg L_3 = \neg L_3 \times L_2\), тогда имеем:\\ \(\Sigma=\{a,b\}\), \(s=11\) и \(T=\{42, 43, 44, 45, 46\}\). Выпишем переходы для \(L_5\): 
        \begin{center}
            \begin{tabular}{ |c|c|c|c|c| } 
                \hline
                \(L_2\) & \(\neg L_3\) & переход по \(a\) & переход по \(b\) \\
                \hline\hline
                1 & 1 & 24 & 22 \\
                \hline 
                1 & 2 & 25 & 23 \\
                \hline
                1 & 3 & 26 & 21 \\
                \hline
                1 & 4 & 21 & 25 \\
                \hline
                1 & 5 & 22 & 26 \\
                \hline
                1 & 6 & 23 & 24 \\
                \hline\hline
                
                2 & 1 & 34 & 32 \\
                \hline
                2 & 2 & 35 & 33 \\
                \hline
                2 & 3 & 36 & 31 \\
                \hline
                2 & 4 & 31 & 35 \\
                \hline
                2 & 5 & 32 & 36 \\
                \hline
                2 & 6 & 33 & 34 \\
                \hline
            \end{tabular} \:\:
            \begin{tabular}{ |c|c|c|c|c| } 
                \hline
                \(L_2\) & \(\neg L_3\) & переход по \(a\) & переход по \(b\) \\
                \hline\hline
                3 & 1 & 44 & 32 \\
                \hline
                3 & 2 & 45 & 33 \\
                \hline
                3 & 3 & 46 & 31 \\
                \hline
                3 & 4 & 41 & 35 \\
                \hline
                3 & 5 & 42 & 36 \\
                \hline
                3 & 6 & 43 & 34 \\
                \hline\hline
                
                4 & 1 & 34 & 32 \\
                \hline
                4 & 2 & 35 & 33 \\
                \hline
                4 & 3 & 36 & 31 \\
                \hline
                4 & 4 & 31 & 35 \\
                \hline
                4 & 5 & 32 & 36 \\
                \hline
                4 & 6 & 33 & 34 \\
                \hline
            \end{tabular}
        \end{center} 
        Построим автомат:\\
        \begin{center}
            \digraph[scale=0.5]{g253}{
            	rankdir=LR; 
            	node [shape=none]; 0;
            	node [shape=doublecircle]; 42 43 44 45 46;
            	node [shape=circle]; 11 12 13 14 15 16 21 22 23 24 25 26 31 32 33 34 35 36 41;
            	0 -> 11;
            	11 -> 24 [label="a"]
            	12 -> 25 [label="a"]
            	13 -> 26 [label="a"]
            	14 -> 21 [label="a"]
            	15 -> 22 [label="a"]
            	16 -> 23 [label="a"]
            	21 -> 34 [label="a"]
            	22 -> 35 [label="a"]
            	23 -> 36 [label="a"]
            	24 -> 31 [label="a"]
            	25 -> 32 [label="a"]
            	26 -> 33 [label="a"]
            	31 -> 44 [label="a"]
            	32 -> 45 [label="a"]
            	33 -> 46 [label="a"]
            	34 -> 41 [label="a"]
            	35 -> 42 [label="a"]
            	36 -> 43 [label="a"]
            	41 -> 34 [label="a"]
            	42 -> 35 [label="a"]
            	43 -> 36 [label="a"]
            	44 -> 31 [label="a"]
            	45 -> 32 [label="a"]
            	46 -> 33 [label="a"]
            
            	11 -> 22 [label="b"]
            	12 -> 23 [label="b"]
            	13 -> 21 [label="b"]
            	14 -> 25 [label="b"]
            	15 -> 26 [label="b"]
            	16 -> 24 [label="b"]
            	21 -> 32 [label="b"]
            	22 -> 33 [label="b"]
            	23 -> 31 [label="b"]
            	24 -> 35 [label="b"]
            	25 -> 36 [label="b"]
            	26 -> 34 [label="b"]
            	31 -> 42 [label="b"]
            	32 -> 43 [label="b"]
            	33 -> 41 [label="b"]
            	34 -> 45 [label="b"]
            	35 -> 46 [label="b"]
            	36 -> 44 [label="b"]
            	41 -> 32 [label="b"]
            	42 -> 33 [label="b"]
            	43 -> 31 [label="b"]
            	44 -> 35 [label="b"]
            	45 -> 36 [label="b"]
            	46 -> 34 [label="b"]
            }
        \end{center}
    \end{enumerate}
    
    \section*{Задание 3}
    Построить минимальные ДКА по регулярным выражениям:
    
    \begin{enumerate}
        \item \((ab + aba)^{*}a\) \\
        Составим недетерминированный автомат, чтобы затем преобразовать его в детерминированный:
        \begin{center}
            \digraph[scale=0.5]{g311}{
                node [shape = none]; 0;
                node [shape = doublecircle]; 2;
                node [shape = circle];
                rankdir=LR; 
                0 -> 1;
                1 -> 2 [label = "a"];
                1 -> 3 [label = "a"];
                1 -> 4 [label = "a"];
                3 -> 1 [label = "b"];
                4 -> 5 [label = "b"];
                5 -> 1 [label = "a"];
            }  
        \end{center}
        Построим эквивалентный ДКА по алгоритму Томпсона:\\
        \begin{center}
             \begin{tabular}{ |c|c|c| } 
                \hline
                \(Q\) & \(a\) & \(b\) \\
                \hline\hline
                1 & 234 & - \\
                \hline
                234 & - & 15 \\
                \hline
                15 & 1234 & - \\
                \hline
                1234 & 234 & 15 \\
                \hline
            \end{tabular}
        \end{center}
        Получили минимальный и детерминированный автомат:
        \begin{center}
            \digraph[scale=0.5]{g313}{
                node [shape = none]; 0;
                node [shape = doublecircle]; 234 1234;
                node [shape = circle];
                rankdir=LR; 
                0 -> 1;
                1 -> 234 [label = "a"];
                234 -> 15 [label = "b"];
                15 -> 1234 [label = "a"];
                1234 -> 234 [label = "b"];
                1234 -> 15 [label = "b"];
            }  
        \end{center}
        
        \item \(a(a(ab)^{*}b)^{*}(ab)^{*}\)\\
        Построим НКА:
        \begin{center}
            \digraph[scale=0.5]{g321}{
                node [shape = none]; 0;
                node [shape = doublecircle]; 2 5;
                node [shape = circle];
                rankdir=LR; 
                0 -> 1;
                1 -> 2 [label = "a"];
                2 -> 3 [label = "a"];
                2 -> 5 [label = "a"];
                3 -> 4 [label = "a"];
                4 -> 3 [label = "b"];
                3 -> 2 [label = "b"];
                5 -> 2 [label = "b"];
            }  
        \end{center}
        Построим эквивалентный ДКА по алгоритму Томпсона:\\
        \begin{center}
            \begin{tabular}{ |c|c|c| } 
                \hline
                \(Q\) & \(a\) & \(b\) \\
                \hline\hline
                1 & 2 & - \\
                \hline
                2 & 35 & - \\
                \hline
                35 & 4 & 2 \\
                \hline
                4 & - & 3 \\
                \hline
                3 & 4 & 2 \\
                \hline
            \end{tabular}
        \end{center}
        Получили следующий детерминированный автомат:
        \begin{center}
            \digraph[scale=0.5]{g322}{
                node [shape = none]; 0;
                node [shape = doublecircle]; 2;
                node [shape = circle];
                rankdir=LR; 
                0 -> 1;
                1 -> 2 [label = "a"];
                2 -> 35 [label = "a"];
                35 -> 4 [label = "a"];
                35 -> 2 [label = "b"];
                4 -> 3 [label = "b"];
                3 -> 4 [label = "a"];
                3 -> 2 [label = "b"];
            }  
        \end{center}
        Определим пары различимых состояний и попробуем минимизировать полученный автомат:
        \begin{center}
            \begin{tabular}{ |c||c|c|c|c|c|c| } 
                \hline
                  & 1 & 2 & 3 & 35 & 4 \\
                \hline\hline
                1  &   & + & + & + & + \\
                \hline
                2  & + &  & + & + & + \\
                \hline
                3  & + & + &  &  & + \\
                \hline
                35 & + & + &  &  & + \\
                \hline
                4 & + & + & + & + &  \\
                \hline
            \end{tabular}
        \end{center}
        Получили различимые состояния \(\{1, 2, 3, 335, 4\}\), перестроим автомат, очевидно, что он минимальный:
        \begin{center}
            \digraph[scale=0.5]{g323}{
                node [shape = none]; 0;
                node [shape = doublecircle]; 2;
                node [shape = circle];
                rankdir=LR; 
                0 -> 1;
                1 -> 2 [label = "a"];
                2 -> 335 [label = "a"];
                335 -> 4 [label = "a"];
                335 -> 2 [label = "b"];
                4 -> 335 [label = "b"];
            }  
        \end{center}
        
        \item \((a + (a + b)(a + b)b)^*\)\\
        Построим НКА:
        \begin{center}
            \digraph[scale=0.5]{g341}{
                node [shape = none]; 0;
                node [shape = doublecircle]; 1;
                node [shape = circle];
                rankdir=LR; 
                0 -> 1;
                1 -> 1 [label = "a"];
                1 -> 2 [label = "a, b"];
                2 -> 3 [label = "a, b"];
                3 -> 1 [label = "b"];
            }  
        \end{center}
        Построим эквивалентный ДКА по алгоритму Томпсона:\\
        \begin{center}
            \begin{tabular}{ |c|c|c| } 
                \hline
                \(Q\) & \(a\) & \(b\) \\
                \hline\hline
                1 & 12 & 2 \\
                \hline
                12 & 123 & 23 \\
                \hline
                2 & 3 & 3 \\
                \hline
                123 & 123 & 123 \\
                \hline
                23 & 3 & 13 \\
                \hline
                3 & - & 1 \\
                \hline
                13 & 12 & 12 \\
                \hline
            \end{tabular}
        \end{center}
        Получили следующий детерминированный автомат:
        \begin{center}
            \digraph[scale=0.5]{g342}{
                node [shape = none]; 0;
                node [shape = doublecircle]; 1 12 13 123;
                node [shape = circle];
                rankdir=LR; 
                0 -> 1;
                1 -> 12 [label = "a"];
                1 -> 2 [label = "b"];
                12 -> 123 [label = "a"];
                12 -> 23 [label = "b"];
                2 -> 3 [label = "a, b"];
                123 -> 123 [label = "a, b"];
                23 -> 3 [label = "a"];
                23 -> 13 [label = "b"];
                3 -> 1 [label = "b"];
                13 -> 12 [label = "a, b"];
            }  
        \end{center}
        Определим пары различимых состояний и попробуем минимизировать полученный автомат:
        \begin{center}
            \begin{tabular}{ |c||c|c|c|c|c|c|c|c|c| } 
                \hline
                    & 1 & 12 & 13 & 123 & 2 & 23 & 3 \\
                \hline\hline
                1   &   &   &     &     & + &  + & + \\
                \hline
                12  &   &   &     &     & + &  + & + \\
                \hline
                13  &   &   &     &     & + &  + & + \\
                \hline
                123 &   &   &     &     & + &  + & + \\
                \hline
                2   & + & + &  +  &  +  &   &    & + \\
                \hline
                23  & + & + &  +  &  +  &   &    & + \\
                \hline
                3   & + & + &  +  &  +  & + & +  &   \\
                \hline
            \end{tabular}
        \end{center}
        
        \item \((b+c)((ab)^*c + (ba)^*)^*\)\\
        Сразу можем построить ДКА:
        \begin{center}
            \digraph[scale=0.5]{g331}{
                node [shape = none]; 0;
                node [shape = doublecircle]; 2 5 7;
                node [shape = circle];
                rankdir=LR; 
                0 -> 1;
                1 -> 2 [label = "b, c"];
                2 -> 3 [label = "a"];
                2 -> 6 [label = "b"];
                3 -> 4 [label = "b"];
                4 -> 5 [label = "c"];
                4 -> 3 [label = "a"];
                5 -> 3 [label = "a"];
                5 -> 6 [label = "b"];
                6 -> 7 [label = "a"];
                7 -> 3 [label = "a"];
                7 -> 6 [label = "b"];
            }  
        \end{center}
        Определим пары различимых состояний и попробуем минимизировать полученный автомат:
        \begin{center}
            \begin{tabular}{ |c||c|c|c|c|c|c|c|c|c| } 
                \hline
                  & 1 & 2 & 3 & 4 & 5 & 6 & 7 \\
                \hline\hline
                1 &   & + & + & + & + & + & + \\
                \hline
                2 & + &   & + & + &   & + &   \\
                \hline
                3 & + & + &   & + & + & + & + \\
                \hline
                4 & + & + & + &   & + & + & + \\
                \hline
                5 & + &   & + & + &   & + &   \\
                \hline
                6 & + & + & + & + & + &   & + \\
                \hline
                7 & + &   & + & + &   & + &   \\
                \hline
            \end{tabular}
        \end{center}
        Получили различимые состояния \(\{1, 257, 3, 4, 5, 6\}\),  перестроим автомат, очевидно, что он минимальный:
        \begin{center}
            \digraph[scale=0.5]{g332}{
                node [shape = none]; 0;
                node [shape = doublecircle]; 257;
                node [shape = circle];
                rankdir=LR; 
                0 -> 1;
                1 -> 257 [label = "b, c"];
                257 -> 3 [label = "a"];
                257 -> 6 [label = "b"];
                3 -> 4 [label = "b"];
                4 -> 257 [label = "c"];
                4 -> 3 [label = "a"];
                6 -> 257 [label = "a"];
            }  
        \end{center}
        
        \item \((a+b)^+(aa + abab + bb + baba)(a+b)^+\)\\
        Построим НКА:
        \begin{center}
            \digraph[scale=0.5]{g351}{
                node [shape = none]; 0;
                node [shape = doublecircle]; 13;
                node [shape = circle];
                rankdir=LR; 
                0 -> 1;
                1 -> 2 [label = "a, b"];
                2 -> 1 [label = "a, b"];
                2 -> 3 [label = "a"];
                2 -> 8 [label = "b"];
                3 -> 4 [label = "a"];
                3 -> 5 [label = "b"];
                4 -> 13 [label = "a, b"];
                5 -> 6 [label = "a"];
                6 -> 7 [label = "b"];
                7 -> 13 [label = "a, b"];
                8 -> 9 [label = "b"];
                8 -> 10 [label = "a"];
                9 -> 13 [label = "a, b"];
                10 -> 11 [label = "b"];
                11 -> 12 [label = "a"];
                12 -> 13 [label = "a, b"];
            }  
        \end{center}
         Построим эквивалентный ДКА по алгоритму Томпсона:\\
        \begin{center}
            \begin{tabular}{ |c|c|c| } 
                \hline
                \(Q\) & \(a\) & \(b\) \\
                \hline\hline
                1 & 2 & 2 \\
                \hline
                2 & 13 & 18 \\
                \hline
                13 & 3 & 3 \\
                \hline
                123 & 123 & 123 \\
                \hline
                23 & 3 & 13 \\
                \hline
                3 & - & 1 \\
                \hline
                13 & 12 & 12 \\
                \hline
            \end{tabular}
        \end{center}

    \end{enumerate} 
    
    \section*{Задание 4}
    Определить, является ли следующие языки регулярными или нет:

    \begin{enumerate}
        \item \(L=\{(aab)^{n}b(aba)^{m} : n \geqslant 0, m \geqslant 0\}\)\\
        Язык регулярный, так как по нему возможно составить ДКА:
        \begin{center}
           \digraph[scale=0.5]{g41}{
                node [shape = none]; 0;
                node [shape = doublecircle]; 5 8;
                node [shape = circle];
                rankdir=LR; 
                0 -> 1;
                1 -> 2 [label = "a"];
                1 -> 5 [label = "b"];
                2 -> 3 [label = "a"];
                3 -> 4 [label = "b"];
                4 -> 5 [label = "b"];
                4 -> 2 [label = "a"];
                5 -> 6 [label = "a"]; 
                6 -> 7 [label = "b"];
                7 -> 8 [label = "a"];
                8 -> 5 [label = "a"];
            }  
        \end{center}
    
        \item \(L = \{uaav : u \in \{a, b\}^*, \; v \in \{a, b\}^*, |u|_b \geqslant |v|_a\}\)\\
        Применим лемму о разрастании. Зафиксируем \(\forall n \in \mathbb{N} \) и рассмотрим слово \(\omega = b^{n}aaa^{n}, \; |\omega| = 2n + 2 \geq n\). Теперь рассмотрим все разбиения этого слова \(\omega = xyz\) такие, что \(|y| \neq 0, \; |xy| \leq n\):
        $$x = b^{k}, \; y = b^{l}, \; z = b^{n - k - l}aaa^n,$$ \begin{center}
            где \(1 \leq k + l \leq n \; \wedge \; l > 0\)
        \end{center} 
        Дргуих разбиенний, удовлетворяющих данным условиям, нет.
        Для любого из таких разбиений слово \(xy^0z \notin L\). Лемма не выполняется, значит, \(L\) не регулярный язык.
        
        \item \(L = \{a^mw : w \in \{a, b\}^{*}, \; 1 \geqslant |w|_b \geqslant m\}\)\\
        Применим лемму о разрастании. Зафиксируем \(\forall n \in \mathbb{N} \) и рассмотрим слово \(\omega = a^nb^n, \; |\omega| = 2n \geqslant n\). Теперь рассмотрим все разбиения этого слова \(\omega = xyz\) такие, что \(|y| \neq 0, \; |xy| \leq n\):
        $$x = a^{l}, \; y = a^{m}, \; z = a^{n-l-m}b^{n},$$ 
        \begin{center}
            где \(l + k \leqslant n \; \wedge \; m \ne 0\)
        \end{center} 
        Дргуих разбиенний, удовлетворяющих данным условиям, нет. Теперь выполним накачку: 
        $$xy^{i}z = a^{l}(a^{m})^{i}a^{n-l-m}b^{n} = a^{n-mi}b^{n} \notin L, \; i 
        \geqslant 0 \in \mathbb{N} $$
        Лемма не выполняется, значит, \(L\) не регулярный язык.
        
        \item \(L = \{a^{k}b^{m}a^{n} : k = n \vee m > 0\}\)\\
        Применим лемму о разрастании. Зафиксируем \(\forall n \in \mathbb{N} \) и рассмотрим слово \(\omega = a^nba^n, \; |\omega| = 2n + 1 \geqslant n\). Теперь рассмотрим все разбиения этого слова \(\omega = xyz\) такие, что \(|y| \neq 0, \; |xy| \leq n\):
        $$x = a^{k}, \; y = a^{m}, \; z = a^{n-k-m}ba^{n},$$ 
        \begin{center}
            где \(k + m \leqslant n \; \wedge \; m \ne 0\)
        \end{center} 
        Дргуих разбиенний, удовлетворяющих данным условиям, нет.
        Теперь выполним накачку: 
        $$xy^{i}z = a^{k}(a^{m})^{i}a^{n-k-m}ba^{n} = a^{n+m(i-1)}ba^{n} \notin L, \; i 
        \geqslant 2 \in \mathbb{N} $$
        Получили противоречие, лемма не выполняется, значит, \(L\) не регулярный язык.
        
        \item \(L = \{ucv : u \in \{a, b\}^*, \; v \in \{a, b\}^*, u \ne v^R \}\)\\
        Применим лемму о разрастании. Зафиксируем \(\forall n \in \mathbb{N} \) и рассмотрим слово \(\omega = (ab)^nc(ab)^n = \alpha_1\alpha_2...\alpha_{4n+1}, \; |\omega| = 4n + 1 \geqslant n\). Теперь рассмотрим все разбиения этого слова \(\omega = xyz\) такие, что \(|y| \neq 0, \; |xy| \leq n\):
        $$x = \alpha_1\alpha_2...\alpha_k, \; y = \alpha_{k+1}...\alpha_{k+m}, \; z = \alpha_{k+m+1}...\alpha_{4n+1}c(ab)^n,$$
        \begin{center}
            где \(k + m \leqslant n \; \wedge \; m \ne 0\)
        \end{center} 
        Дргуих разбиенний, удовлетворяющих данным условиям, нет. Теперь выполним накачку: 
        $$xy^{i}z = (\alpha_1\alpha_2...\alpha_k)(\alpha_{k+1}...\alpha_{k+m})^i(\alpha_{k+m+1}...\alpha_{4n+1}c(ab)^n)$$
        При \(i = 2\) имеем:
        $$xy^{2}z = (\alpha_1\alpha_2...\alpha_k)(\alpha_{k+1}...\alpha_{k+m})^2(\alpha_{k+m+1}...\alpha_{4n+1}c(ab)^n) \notin L$$
        Лемма не выполняется, значит, \(L\) не регулярный язык.
    
    \end{enumerate}
    
    \section*{Задние 5}
    Реализовать алгоритмы:
    \begin{enumerate}
        \item Построение ДКА по НКА с \(\lambda\)-переходами (не сделал).
        \item Прямое произведение языков, с возможностью построить пересечение, объединение и разность.\\
        Реализовал алгоритмы, добавил возможность созранения графов в виде картинок и создания pdf-файла с решением для вычисления одного действия над двумя автоматами. \\Все файлы, относящиеся к заданию приложил в репозиторий:
        \begin{itemize}
            \item automaton.py - модуль с реализацией модели ДКА, необходимых алгоритмов.
            \item solution\_maker.py - модуль для построения решения.
            \item main.py - модуль, в котором демонстрируется пример использования разработанных библиотек.
            \item дополнительные pdf-файлы с примером работы.
        \end{itemize}

    \end{enumerate}
     

\end{document}
