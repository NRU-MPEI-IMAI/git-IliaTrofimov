\documentclass{article}
\usepackage[T2A]{fontenc}
\usepackage[utf8]{inputenc}
\usepackage[russian]{babel}
\usepackage{textcomp}
\usepackage{color}
\usepackage{xspace}
\usepackage{multirow}
\usepackage{amsmath,amsfonts,amsthm,amssymb,amsbsy,amstext,amscd,amsxtra,multicol}
\usepackage{indentfirst}
\usepackage{verbatim}
\usepackage[left=2cm, right=2cm, top=2cm, bottom=2cm, bindingoffset=0cm]{geometry}

\usepackage[pdf]{graphviz}
\usepackage{morewrites}

\usepackage{xpatch}
\makeatletter
\newcommand*{\addFileDependency}[1]{% argument=file name and extension
  \typeout{(#1)}
  \@addtofilelist{#1}
  \IfFileExists{#1}{}{\typeout{No file #1.}}
}
\makeatother
\xpretocmd{\digraph}{\addFileDependency{#2.dot}}{}{}


\author{Трофимов И.С., группа А-05-19}
\title{Домашняя работа по ТМВ №1}

\begin{document}
    \maketitle
    
    \section*{Задание 1}
    Построить конечные автоматы, распознающие следующие языки\:
    \begin{enumerate}
        \item \(L_1=\{\omega\in\{a,b,c\}^* : |\omega|_c = 1 \} \)
        \begin{center}
            \digraph[scale=0.5]{g11}{
                node [shape = none]; 0;
                node [shape = doublecircle]; 2;
    	        node [shape = circle];
                rankdir=LR; 
                0 -> 1;
                1 -> 2 [label = "c"];
                1 -> 1 [label = "a,b"];
                2 -> 2 [label = "a,b"];
            }   
        \end{center}
        \item \(L_2=\{\omega\in\{a,b\}^* : |\omega|_a \leqslant 2, |\omega|_b \geqslant 2 \} \) \\
        Рассмотрим автоматы \(A=\{\omega\in\{a,b\}^* : |\omega|_a \leqslant 2 \} \) и \(B=\{\omega\in\{a,b\}^* : |\omega|_B \geqslant 2 \} \), распознающиие каждое условие по отдельности:
        \begin{center}
            \digraph[scale=0.5]{g121}{
                node [shape = doublecircle]; 1 2 3;
                node [shape = none]; 0;
                rankdir=LR; 
                0 -> 1;
                1 -> 2 [label = "a"];
                2 -> 3 [label = "a"];
                1 -> 1 [label = "b"];
                2 -> 2 [label = "b"];
                3 -> 3 [label = "b"];
            }
            \digraph[scale=0.5]{g122}{
                node [shape = doublecircle]; 3;
                node [shape = none]; 0;
                node [shape = circle]; 1 2;
                rankdir=LR; 
                0 -> 1;
                1 -> 2 [label = "b"];
                2 -> 3 [label = "b"];
                1 -> 1 [label = "a"];
                2 -> 2 [label = "a"];
                3 -> 3 [label = "a,b"];
            }
        \end{center}
        Тогда \(L_2=A \times B\). Терминальными состояниями в \(L_2\) будут вершины 13, 23 и 33. Теперь выпишем переходы для произведения автоматов в виде таблицы:
        \begin{center}
            \begin{tabular}{ |c|c|c|c| } 
                \hline
                \(A\) & \(B\) & переход по \(a\) & переход по \(b\) \\
                \hline
                1 & 1 & 21 & 12 \\
                \hline
                2 & 2 & 32 & 23 \\
                \hline
                3 & 3 & - & 33 \\
                \hline
                1 & 2 & 22 & 13 \\
                \hline
                2 & 3 & 33 & 23 \\
                \hline
                3 & 1 & - & 32 \\
                \hline
                1 & 3 & 23 & 13 \\
                \hline
                2 & 1 & 31 & 22 \\
                \hline
                3 & 2 & - & 33 \\
                \hline
                \end{tabular}
            \end{center}
        После прямого произведения двух автоматов получим окончательный ответ:
        \begin{center}
            \digraph[scale=0.5]{g123}{
                node [shape = none]; 0;
                node [shape = doublecircle]; 13 23 33;
    	        node [shape = circle];
                rankdir=LR; 
                0 -> 11;
                11 -> 21 [label = "a"];
                11 -> 12 [label = "b"]; 
                22 -> 32 [label = "a"];
                22 -> 23 [label = "b"];
                33 -> 33 [label = "b"];
            	12 -> 22 [label = "a"];
                12 -> 13 [label = "b"];
            	23 -> 33 [label = "a"];
                23 -> 23 [label = "b"];
                31 -> 32 [label = "b"];
            	13 -> 23 [label = "a"];
                13 -> 13 [label = "b"];
            	21 -> 31 [label = "a"];
                21 -> 22 [label = "b"];
                32 -> 33 [label = "b"];
            }
        \end{center}
        
        \item \(L_3=\{\omega\in\{a,b\}^*:|\omega|_a \neq |\omega|_b \}\) \\
        Этот язык нельзя описать с помощью ДКА, т.к. для описания языка необходимо запоминать количество символов одного типа, что ДКА сделать не может.
        \item \(L_4=\{\omega\in\{a,b\}^* : \omega \omega = \omega \omega \omega \} \) \\
        Очевидно, что такой язык описывает только пустые слова:
        \begin{center}
            \digraph[scale=0.5]{g14}{
                node [shape = none]; 0;
                node [shape = circle]; 1;
                rankdir=LR; 
                0 -> 1;
                1 -> 1 [label = "a,b"];
            }  
        \end{center}
    \end{enumerate}
    
    \section*{Задание 2}
    Построить конечные автоматы, распознающие слудеющие языки, используя прямое произведение:
    \begin{enumerate}
        \item \(L_1=\{\omega\in\{a,b\}^* : |\omega|_a \geqslant 2 \wedge |\omega|_b
        \geqslant 2 \} \) \\
        Рассмотрим автоматы \(A=\{\omega\in\{a,b\}^* : |\omega|_a \geqslant 2 \} \) и \(B=\{\omega\in\{a,b\}^* : |\omega|_b \geqslant 2 \} \), распознающиие каждое условие по отдельности:
        \begin{center}
            \digraph[scale=0.5]{g211}{
                node [shape = none]; 0;
                node [shape = doublecircle]; 3;
                node [shape = circle]; 1 2;
                rankdir=LR; 
                0 -> 1;
                1 -> 2 [label = "a"];
                2 -> 3 [label = "a"];
                1 -> 1 [label = "b"];
                2 -> 2 [label = "b"];
                3 -> 3 [label = "a,b"];
            }
            \digraph[scale=0.5]{g212}{
                node [shape = none]; 0;
                node [shape = doublecircle]; 3;
                node [shape = circle]; 1 2;
                rankdir=LR; 
                0 -> 1;
                1 -> 2 [label = "b"];
                2 -> 3 [label = "b"];
                1 -> 1 [label = "a"];
                2 -> 2 [label = "a"];
                3 -> 3 [label = "a,b"];
            }    
        \end{center}
        Тогда \(L_1=A \times B\), имеем \(\Sigma=\{a,b\}\), \(s=11\) и \(T=\{33\}\). Теперь выпишем переходы для произведения автоматов в виде таблицы:
        \begin{center}
            \begin{tabular}{ |c|c|c|c| } 
                \hline
                \(A\) & \(B\) & переход по \(a\) & переход по \(b\) \\
                \hline
                1 & 1 & 21 & 12 \\
                \hline
                2 & 2 & 32 & 23 \\
                \hline
                3 & 3 & 33 & 33 \\
                \hline
                1 & 2 & 22 & 13 \\
                \hline
                2 & 3 & 33 & 23 \\
                \hline
                3 & 1 & 31 & 32 \\
                \hline
                1 & 3 & 23 & 13 \\
                \hline
                2 & 1 & 31 & 22 \\
                \hline
                3 & 2 & 32 & 33 \\
                \hline
            \end{tabular}
        \end{center}
        После прямого произведения двух автоматов получим окончательный ответ:
        \begin{center}
            \digraph[scale=0.5]{g213}{
                node [shape = none]; 0;
                node [shape = doublecircle]; 33;
    	        node [shape = circle];
                rankdir=LR; 
                0 -> 11;
                11 -> 21 [label = "a"];
                11 -> 12 [label = "b"]; 
                22 -> 32 [label = "a"];
                22 -> 23 [label = "b"];
                33 -> 33 [label = "a,b"];
                
            	12 -> 22 [label = "a"];
                12 -> 13 [label = "b"];
            	23 -> 33 [label = "a"];
                23 -> 23 [label = "b"];
                31 -> 31 [label = "a"];
                31 -> 32 [label = "b"];
                
            	13 -> 23 [label = "a"];
                13 -> 13 [label = "b"];
            	21 -> 31 [label = "a"];
                21 -> 22 [label = "b"];
                32 -> 32 [label = "a"];
                32 -> 33 [label = "b"];
            }
        \end{center}
        
        \item \(L_2=\{\omega \in\{a,b\}^* : |\omega| \geqslant 3 \wedge |\omega| \text{ нечётное} \} \) \\
        Рассмотрим автоматы \(A=\{\omega \in\{a,b\}^* : |\omega| \geqslant 3\} \) и \(B=\{\omega \in\{a,b\}^* : |\omega| \text{ нечётное} \} \):
        \begin{center}
            \digraph[scale=0.5]{g221}{
                node [shape = none]; 0;
                node [shape = doublecircle]; 4;
                node [shape = circle]; 1 2, 3;
                rankdir=LR; 
                0 -> 1;
                1 -> 2 [label = "a, b"];
                2 -> 3 [label = "a, b"];
                3 -> 4 [label = "a, b"];
                4 -> 4 [label = "a, b"];
            }
            \digraph[scale=0.5]{g212}{
                node [shape = none]; 0;
                node [shape = doublecircle]; 2;
                node [shape = circle]; 1;
                rankdir=LR; 
                0 -> 1;
                1 -> 2 [label = "a, b"];
                2 -> 1 [label = "a, b"];
            }    
        \end{center}
        Тогда \(L_2=A \times B\), имеем \(\Sigma=\{a,b\}\), \(s=11\) и \(T=\{33\}\). Переходы для произведения автоматов:
        \begin{center}
            \begin{tabular}{ |c|c|c| } 
                \hline
                \(A\) & \(B\) & переход по \(a\) или \(b\) \\
                \hline
                1 & 1 & 22 \\
                \hline
                2 & 1 & 32 \\
                \hline
                3 & 1 & 42 \\
                \hline
                4 & 1 & 42 \\
                \hline
                1 & 2 & 21 \\
                \hline
                2 & 2 & 31 \\
                \hline
                3 & 2 & 41 \\
                \hline
                4 & 2 & 41 \\
                \hline
            \end{tabular}
        \end{center}
        После прямого произведения двух автоматов получим окончательный ответ:
        \begin{center}
            \digraph[scale=0.5]{g223}{
                node [shape = none]; 0;
                node [shape = doublecircle]; 42;
    	        node [shape = circle];
                rankdir=LR; 
                0 -> 11;
                11 -> 22 [label = "a,b"];
                21 -> 32 [label = "a,b"];
                31 -> 42 [label = "a,b"];
                41 -> 42 [label = "a,b"];
                12 -> 21 [label = "a,b"];
                22 -> 31 [label = "a,b"];
                32 -> 41 [label = "a,b"];
                42 -> 41 [label = "a,b"];
            }
        \end{center}
        ДКА можно упростить, т.к. невозможно попасть в узлы 12, 21 и 32:
        \begin{center}
            \digraph[scale=0.5]{g224}{
                node [shape = none]; 0;
                node [shape = doublecircle]; 42;
    	        node [shape = circle];
                rankdir=LR; 
                0 -> 11;
                11 -> 22 [label = "a,b"];
                31 -> 42 [label = "a,b"];
                41 -> 42 [label = "a,b"];
                22 -> 31 [label = "a,b"];
                42 -> 41 [label = "a,b"];
            }
        \end{center}
        С другой стороны, описать данный язык можно с помощью более компактного автомата, созданного "вручную":
        \begin{center}
            \digraph[scale=0.5]{g225}{
                node [shape = none]; 0;
                node [shape = doublecircle]; 4;
    	        node [shape = circle];
                rankdir=LR; 
                0 -> 1;
                1 -> 2 [label = "a,b"];
                2 -> 3 [label = "a,b"];
                3 -> 4 [label = "a,b"];
                4 -> 3 [label = "a,b"];
            }
        \end{center}
        
        \item \(L_3=\{\omega \in\{a,b\}^* : |\omega|_a \text{ чётно} \wedge |\omega|_b \text{ кратно } 3 \} \) \\
        Рассмотрим автоматы \(A=\{\omega \in\{a,b\}^* : |\omega|_a \text{ чётно} \} \) и \(B=\{\omega \in\{a,b\}^* : |\omega|_b \text{ кратно } 3 \} \):
        \begin{center}
            \digraph[scale=0.5]{g231}{
                node [shape = none]; 0;
                node [shape = doublecircle]; 1;
    	        node [shape = circle];
                rankdir=LR; 
                0 -> 1;
                1 -> 2 [label = "a"];
                2 -> 1 [label = "a"];
                1 -> 1 [label = "b"];
                2 -> 2 [label = "b"];
            }
            \digraph[scale=0.5]{g232}{
                node [shape = none]; 0;
                node [shape = doublecircle]; 1;
    	        node [shape = circle];
                rankdir=LR; 
                0 -> 1;
                1 -> 1 [label = "a"];
                2 -> 2 [label = "a"];
                3 -> 3 [label = "a"];
                1 -> 2 [label = "b"];
                2 -> 3 [label = "b"];
                3 -> 1 [label = "b"];
            }
        \end{center}
        Тогда \(L_3=A \times B\), имеем \(\Sigma=\{a,b\}\), \(s=11\) и \(T=\{11\}\). Переходы для произведения автоматов:
        \begin{center}
            \begin{tabular}{ |c|c|c|c| } 
                \hline
                \(A\) & \(B\) & переход по \(a\) & переход по \(b\) \\
                \hline
                1 & 1 & 21 & 12 \\
                \hline
                2 & 2 & 12 & 23 \\
                \hline
                1 & 2 & 22 & 13 \\
                \hline
                2 & 3 & 13 & 21 \\
                \hline
                1 & 3 & 23 & 11 \\
                \hline
                2 & 1 & 11 & 22 \\
                \hline
            \end{tabular}
        \end{center}
        После прямого произведения двух автоматов получим окончательный ответ:
        \begin{center}
            \digraph[scale=0.5]{g233}{
                node [shape = none]; 0;
                node [shape = doublecircle]; 11;
    	        node [shape = circle];
                rankdir=LR; 
                0 -> 11;
                11 -> 21 [label = "a"];
                11 -> 12 [label = "b"];
                22 -> 12 [label = "a"];
                22 -> 23 [label = "b"];
                12 -> 22 [label = "a"];
                12 -> 13 [label = "b"];
                23 -> 13 [label = "a"];
                23 -> 21 [label = "b"];
                13 -> 23 [label = "a"];
                13 -> 11 [label = "b"];
                21 -> 11 [label = "a"];
                21 -> 22 [label = "b"];
            }
        \end{center}
        
        \item \(L_4= \neg L_3\) \\
        Имеем, \(T_4= Q_3 \setminus T_3 = \{ 12, 13, 21, 22, 23 \} \), тогда можно легко построить ДКА:
        \begin{center}
            \digraph[scale=0.5]{g23}{
                node [shape = none]; 0;
                node [shape = doublecircle]; 12 13 21 22 23;
    	        node [shape = circle]; 11;
                rankdir=LR; 
                0 -> 11;
                11 -> 21 [label = "a"];
                11 -> 12 [label = "b"];
                22 -> 12 [label = "a"];
                22 -> 23 [label = "b"];
                12 -> 22 [label = "a"];
                12 -> 13 [label = "b"];
                23 -> 13 [label = "a"];
                23 -> 21 [label = "b"];
                13 -> 23 [label = "a"];
                13 -> 11 [label = "b"];
                21 -> 11 [label = "a"];
                21 -> 22 [label = "b"];
            }
        \end{center}
        
        \item \(L_5= L_2 \setminus L_3\) \\
        Так как \(L_5= L_2 \setminus L_3 = L_2 \cap \neg L_3 = \neg L_3 \times L_2\), тогда имеем:\\ 
        \(\Sigma=\{a,b\}\), \(s=\langle11,11\rangle\) и \(T=\{\langle12,42\rangle, \langle13,42\rangle, \langle21,42\rangle, \langle22,42\rangle,\langle23,42\rangle\}\)\\
        Выпишем переходы для \(L_5\): \begin{center}
            \begin{tabular}{ |c|c|c|c|c| } 
                \hline
                \(\neg L_3\) & \(L_2\) & переход по \(a\) & переход по \(b\) \\
                \hline
                11 & 11 & 21, 22 & 12, 22 \\
                \hline 
                11 & 22 & 21, 31 & 12, 31 \\
                \hline
                11 & 31 & 21, 42 & 12, 42 \\
                \hline
                11 & 42 & 21, 41 & 12, 41 \\
                \hline
                11 & 41 & 21, 42 & 12, 42 \\
                \hline\hline
                12 & 11 & 22, 22 & 13, 22 \\
                \hline
                12 & 22 & 22, 31 & 13, 31 \\
                \hline
                12 & 31 & 22, 42 & 13, 42 \\
                \hline
                12 & 42 & 22, 41 & 13, 41 \\
                \hline
                12 & 41 & 22, 42 & 13, 42 \\
                \hline\hline
                13 & 11 & 23, 22 & 11, 22  \\
                \hline
                13 & 22 & 23, 31 & 11, 31 \\
                \hline
                13 & 31 & 23, 42 & 11, 42 \\
                \hline
                13 & 42 & 23, 41 & 11, 41 \\
                \hline
                13 & 41 & 23, 42 & 11, 42 \\
                \hline\hline
                21 & 11 & 11, 22 & 22, 22 \\
                \hline
                21 & 22 & 11, 31 & 22, 31 \\
                \hline
                21 & 31 & 11, 42 & 22, 42 \\
                \hline
                21 & 42 & 11, 41 & 22, 41 \\
                \hline
                21 & 41 & 11, 42  & 22, 42 \\
                \hline\hline
                22 & 11 & 12, 22 & 23, 22 \\
                \hline
                22 & 22 & 12, 31 & 23, 31 \\
                \hline
                22 & 31 & 12, 42 & 23, 42 \\
                \hline
                22 & 42 & 12, 41 & 23, 41 \\
                \hline
                22 & 41 & 12, 42 & 23, 42 \\
                \hline\hline
                23 & 11 & 13, 22 & 21, 22 \\
                \hline
                23 & 22 & 13, 31 & 21, 31 \\
                \hline
                23 & 31 & 13, 42 & 21, 42 \\
                \hline
                23 & 42 & 13, 41 & 21, 41 \\
                \hline
                23 & 41 & 13, 42 & 21, 42 \\
                \hline
            \end{tabular}
        \end{center} 
        
    \end{enumerate}
     

\end{document}
